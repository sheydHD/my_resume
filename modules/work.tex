\section{Work History}

\subsection*{HIWI, 07/2025 - Current}
\textbf{RWTH Aachen, Lehr- Und Forschungsgebiet Kontinuumsmechanik} - Aachen, Germany
\begin{highlights}
    \item Adjusted \textbf{Deep Symbolic Optimization (DSO)} code (2022) to modern standards using \textbf{Python 3.11} and \textbf{TensorFlow 2.x}.
    \item Increased code functionality extending the framework from \textbf{MISO (Multiple Input Single Output)} to \textbf{MIMO (Multiple Input Multiple Output)}.
    \item Documented codebase and adapted software for compatibility across diverse operating systems.
\end{highlights}

\subsection*{Working Student, 11/2024 - Current}
\textbf{Deutsches Zentrum für Luft- und Raumfahrt (DLR)} - Köln, Germany
\begin{highlights}
    \item Designed and deployed \textbf{DLR-AutoMat}, a full-stack web app (Python/FastAPI backend, React frontend) for automated generation of material cards for LS-Dyna simulations.
    \item Integrated a \textbf{MongoDB-based metadata search engine} to enable fast and efficient retrieval of experimental and simulation data.
    \item Deployed \textbf{Chemotion ELN} with PostgreSQL on company servers and ensured seamless interoperability with DLR-AutoMat and MongoDB.
    \item Connected all systems to the company's \textbf{cloud infrastructure}, \\
ensuring scalability, reliability, and maintainability through Docker and Git-based DevOps workflows.
\end{highlights}

\subsection*{Bachelor Thesis, 06/2024 - 09/2024}
\textbf{Mercedes-Benz Group AG} - Sindelfingen, Germany
\begin{highlights}
    \item Bachelor thesis topic: "Evaluation of ML-based models for the detection of corrosion-prone areas during the vehicle development process"
    \item \textbf{Created and structured datasets} to train and \textbf{evaluate ML models} for identifying corrosion-prone areas during a vehicle design.
    \item Conducted interviews with field experts and further evaluated in-depth performance with applied statistical metric like \textbf{MCC}.
    \item Improved segmentation quality by identifying model weaknesses and refining feature engineering methods in \textbf{MATLAB}.
\end{highlights}

\subsection*{Internship, 10/2023 - 04/2024}
\textbf{Mercedes-Benz Group AG} - Sindelfingen, Germany
\begin{highlights}
    \item Contributed to development of ML-based models for the detection of corrosion-prone areas during vehicle development
    \item Improved segmentation of door components by \textbf{enhancing neural network training datasets} and applying advanced \textbf{feature engineering methods in MATLAB}.
    \item Retrained and optimized models for detecting corrosion protection layers (e.g., PVC, adhesives), \textbf{evaluated using MCC} and custom visualizations.
    \item Developed and integrated algorithms for detection of crucial car structural elements (e.g. hem flange)
\end{highlights}

\subsection*{HIWI, 10/2021 - 10/2023}
\textbf{RWTH Aachen, Verfahrenstechnik} - Aachen, Germany
\begin{highlights}
    \item \textbf{TriggerInk Project} - Surgical Robotic Arm Prototype
    \begin{highlights}
        \item Programmed robotic arm movement for a knee surgery simulation using \textbf{RoboDK} and \textbf{Python}.
        \item Connected the robotic arm and an extruder with an external pump system for precise cartilage extrusion control.
    \end{highlights}
    \item \textbf{Flow, Fouling, and Backwashing with Membrane Filter Modules}
    \begin{highlights}
        \item Designed and conducted experiments on various membrane modules, including AVT-developed hollow fibers and clinical blood filters.
        \item \textbf{Operated MRI equipment to generate high-resolution images} of membrane structures; enhanced image clarity with custom \textbf{MATLAB} scripts.
        \item Contributed to a peer-reviewed \textbf{publication in Journal of Membrane Science}: \\
        \href{https://doi.org/10.1016/j.memsci.2025.124205}{https://doi.org/10.1016/j.memsci.2025.124205}
    \end{highlights}
\end{highlights}
